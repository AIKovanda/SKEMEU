\chapter{[SKE] Odhady spolehlivosti a kumulativní intenzity poruch, typy cenzorování, testy.}
%lecture 10

    \subsection*{Model konkurenčních rizik (Competing risks)}
    
    V této části budeme uvažovat vícero typů poruch $\delta \in \lbrace1,2,\dots k \rbrace$. Nechť je $T$ doba do poruchy. Můžeme si to představit jako sériové zapojení $k$ rizik, které soutěží o vyřazení komponenty, proto Competing risks. To má širokou škálu využití v praxi, $T$ může být například doba trvání manželství, které může končit z mnoha příčin. V ekonomii se může jednat o dobu do krachu firmy. Ve stavitelství se můžeme bavit o době do pádu určitého mostu, apod.
    
    Máme tedy dvojce $(T_j, \delta_j)_{j=1}^n$, $T_j \geq 0$ a $\delta_j\in\widehat{k}$, kde $T_j$ i $\delta_j$ jsou náhodnými veličinami. V ideálních datech pro vyhodnocování by měly být zastoupeny všechny možnosti poruch, jinak není možné takový systém modelovat. Každá dvojce má sdruženou distribuční funkci $(T_j, \delta_j) \sim \FF_{T,\delta}(t,d)$, kde $t\in\R, d\in \R$. Tato distribuční funkce je ovšem v $d$ diskrétní, čili v proměnné $d$ se nachází skoky. Nyní $\FF_{T,\delta}(t,d)$ charakterizujme pomocí subdistribucí
    $$R(t,j) := \PP(T > t, j),\quad t \in \R,~j \in \widehat{k},$$ 
    $$p_j = \PP(\delta = j) = R(0,j),\quad\sum_{i=1}^k p_i = 1.$$ 
    Budeme potřebovat také subhustoty, které definujeme jako $f(t,j)=-R'(t,j)$. Pomocí tohoto objektu lze určit hustota marginální hustota pro $T$, či distribuční a spolehlivostní funkce. Platí tedy, že
    $$f_T(t) = \sum_{i = 1}^{n}f(t,j),\quad\text{a tedy}\qquad\FF_T(t) = \int_{-\infty}^{t} \sum_{j=1}^n f(y,j)\d y = \sum_{j=1}^n \FF(t,j).  $$
    Dále lze odvodit
    $$R_T(t) = \int_{t}^{\infty} \FF(y,j) \d y = \sum_{j=1}^n R(t,j),\qquad r_T(t) = \frac{f_T}{R_T} = \sum_{j=1}^n\frac{f(t,j)}{R_T} = \sum_{j=1}^n r_T(t,j). $$
    Můžeme definovat také relativní intenzitu poruch $j$-tého módu rizika $\frac{r_T(t,j)}{r_T(t)}$.
    Pokračujme definováním dalších veličin,
    $$ R_T(t|j) := \PP(T > t|\delta =j) = \frac{\PP(T > t, \delta =j)}{\PP(\delta =j)} = \frac{R(t,j)}{p_j}, $$
    $$R_T(t) = \sum_{j=1}R(t,j) = \sum_{j=1}p_j R_T(t|j).$$
    V poslední rovnosti jsme získali $R_T(t)$ jako konvexní směs podmíněných distribucí. Tyto vztahy platí i v případě, že jednotlivé módy rizika jsou na sobě závislé.

\section{LIFE-TIME aplikace}

    Budeme využívat \emph{Censoring} dat. To lze udělat několika způsoby.

    \paragraph{Censorování typu I.} Experiment ukončíme v čace $t_c$. Data jsou tvaru $T_{(1)},T_{(2)},\dots, T_{(r)} \leq t_c$, přičemž máme informaci, že $T_{(r+1)},T_{(r+2)},\dots, T_{(n)} > t_c$. To lze zapsat i jako
    $$W_j := \min(T_j,t_c), \quad \delta_j = I\lbrace T_j \leq t_c \rbrace \in\{0,1\}.$$
    Tato úloha by šla řešit přes konkurenční rizika, ale byl by to overkill. V následující větě se podíváme jak se dá sestrojit vztah pro hustotu pravděpodobnosti prvních $r$ uspořádaných statistik, včetně $r$, které je náhodné.
    
    \begin{theorem}
        Nechť $T_1, T_2, \dots, T_n$ jsou iid $\FF$ (nebo $f$). Označme $T_{(1)} < T_{(2)} < \cdots <T_{(r)}$, kde $r\in\widehat{n}$. Pak 
        $$f(t_{(1)}, t_{(2)}, \dots, t_{(n)}, r) = \frac{n!}{(n-r)!}\Big(\prod_{j=1}^r f(t_{(j)})\Big)R(t_c)^{n-r}, \quad  \forall t_{(1)} < t_{(2)} < \cdots <t_{(r)} \leq t_c $$ 
        a navíc
        $$f(t_{(1)}, t_{(2)}, \dots, t_{(n)} |r) = \frac{r!}{\FF^r(t_c)}\prod_{j=1}^r f(t_{(j)}) = r! \prod_{j=1}^r  \frac{f(t_{(j)})}{\FF(t_{c})}, \quad  t_{(j)} \leq t_c,$$ 
        což se dá interpretovat jako useknutá hupsti, která je proto přenormovaná hodnotou $\FF(t_{c})$.
    \end{theorem}

    \paragraph{Censorování typu II.}
    Také se jinak nazývá censorování poruchou. V tomto typu censoringu volíme $r$ pevně a čas experimentu (zde označme $t_{\mathrm{exp}} = t_{(r)}$) je náhodná veličina. Data pak vypadají takto: 
    $$W_j = \min(T_j, t_{(r)})_{j=1}^n,\quad \delta_j = I\lbrace T_j \leq t_{(r)} \rbrace \in \{0,1\}. $$
    I zde můžeme formulovat podobnou větu, jako v případě Censoringu typu I.
    
    \begin{theorem}
        Nechť $T_1, T_2, \dots, T_n$ jsou iid $\FF$ (nebo $f$). Označme $T_{(1)} < T_{(2)} < \cdots <T_{(r)}$, kde $r\in\widehat{n}$. 
        Pak $$f(t_{(1)}, t_{(2)}, \dots, t_{(n)}, r) = \frac{n!}{(n-r)!}\Big(\prod_{j=1}^r f(t_{(j)})\Big)R(t_{(r)})^{n-r}, \quad  \forall t_{(1)} < t_{(2)} < \cdots <t_{(r)} \leq t_c. $$ 
    \end{theorem}
    
    \paragraph{Censorování typu III.}
    
    Kombinace přechozích dvou censorování, tedy nastavíme pevně čas experimentu $t_c$ a také nastavíme pevně $r$. Potom bereme to, co přijde dříve, jestli $r$ poruch, nebo jestli dojde čas.
    
    \paragraph{Random censoring, RC.}
    Máme opět čas do poruchy $T\sim\FF_T$, navíc máme ovšem další náhodnou veličinu $C\sim\FF_C$, kterou nazveme časový cenzor. Obě tyto veličiny jsou vzájemně nezávislé. Časový cenzor realizuje censorování každé jednotlivé časové hodnoty. Data v tomto případě zapisujeme ve tvaru
    $$W = \min(T,C), \quad \delta = I\lbrace T \leq C \rbrace. $$
    Tím dostáváme opět spojitě diskrétní rozdělení pro dvojici
    $(W,\delta)$ v modelu konkurenčních rizik. Tím máme hotový setup modelu. Střední hodnotu potom můžeme vypočítat jako
    $$q = \EE{\delta} =  \EE{I\lbrace T \leq C \rbrace} = \int\limits_{\lbrace T \leq C \rbrace}1\d\PP = \PP(T\leq C)\equal{id} \iint_{t\leq c}\d\FF_T\d\FF_C = \iint_{t\leq c}f_Tf_C  \d t \d c.$$
    Víme-li, že máme k dispozici $n$ pozorování, potom číslo $n\cdot q$ značí střední počet poruch v $n$-tici opakování. \\
    Zaznamenáváme $T_j, C_j$, což vede na data ve tvaru $(W_j,\delta_j)_{j=1}^n$. Dále označme $D$ množinu indexů, kdy opravdu došlo k poruše, čili $$D = \lbrace j\in \widehat{n}: \delta_j =1 \rbrace .$$ Často nazývána také jako množina defektů. Mohutnost množiny $\abs{D} = \# D = \delta = \sum_{j=1}^n\delta_j$, tedy počet poruch v $n$-tici. 
    
    \begin{remark}
        Spolehlivostní funkci $R_T(t)$ v klinických testech většinou značíme jako $S_T(t)$ a je známa pod názvem Survival function. 	
    \end{remark} 
    
    \paragraph{Random censoring zleva.}
    Představme si situaci, kdy doktor musí určit diagnózu, sleduje pacienty, přičemž při již probíhajících testech přijde další pacient. Doktor ho tudíž nesleduje od úplného začátku, ale sleduje ho až od nějakého času $t_{l}$. Někdy známo také jako oboustranné censorování.
    
    \paragraph{RC + censoring typu I.}
    Pracujeme ze živými objekty, které můžou být censorované, ovšem může se stát, že na experiment není dostatek času nebo financí a tudíž musí být experiment ukončen v pevném čase $t_{c}$. Je to tedy smíšené censorování.
    
    \begin{remark}[RC s neúplnou informací.]
    Jedná se o běžný RC, avšak nedokážeme zaznamenat poruchu v tom čase, kdy skutečně nastala. O poruše se dozvíme až na konci celého experimentu. Například pacient onemocnění, ale nepřijde k doktorovi, porucha tudíž skutečně nastala, ale pacient ji neoznámí. Po skončení experimentu doktor obvolá pacienty, že se končí s experiemntem a až v tom okamžiku se pacient přizná, že onemocněl.\\
    Máme tedy jistou pravděpodobnost $p$, že porucha nebyla zachycena a je dostupná až po skončení po experimentu.
    \end{remark}
    
    \section{Výpočty v RC modelech.}
    Začneme větou, která nám prozradí, jak vypadá sdružená hustota veličiny $(W, \delta)$.
    \begin{theorem}
        $W$ je náhodná veličina vzhledem k běžné spojité Lebesgueově míře $\lambda$ a náhodná veličina $\delta$ je náhodná veličina vzhledem k čítací míře $\kappa$. Dále jsou $T$ a $C$ nezávislé, přičemž máme k dispozici $f_T$ a $f_C$. Pak platí
        $$f(w,\delta) =  \Big[f_T(w)(1-F_C(w))\Big] ^{\delta}\Big[f_C(w)(1-F_T(w)) \Big]^{1-\delta}$$
        pro $\forall \delta\in\lbrace 0,1\rbrace$ a $\forall w\in\R^{+}$ vzhledem k součinové míře $\lambda \times \kappa$.
    \end{theorem}

    \begin{dusl}
        $\FF_T = \FF_T(t,\boldsymbol{\theta}_1)$ a $\FF_C = \FF_C(c,\boldsymbol{\theta}_2)$ a nechť $\boldsymbol{\theta}_1$ a $\boldsymbol{\theta}_2$ nejsou provázány. Navíc máme $(W_j,\delta_j)_{j=1}^n$ iid s distribuční funkcí $F_{W,\delta}$. Pak věrohodnostní funkce
        $$L(\boldsymbol{\theta_1}, \boldsymbol{\theta_2}) = L_1(\boldsymbol{\theta_1})L_2(\boldsymbol{\theta_2}) = \Big[\prod_{j\in D }f_T\prod_{j \in \widehat{n}\smallsetminus D}(1-F_T)\Big]\cdot\Big[\prod_{j\in D }(1-F_C)\prod_{j \in \widehat{n}\smallsetminus D}f_C\Big]$$
    \end{dusl}
    \begin{example}
        $T \sim \Exp(\lambda), C\sim  \FF_C(\boldsymbol{\theta_2})$. Pro jednodušší zápis označme $\widehat{n}\smallsetminus D = U$ a připomeňme, že $\delta = \abs{D}$.  Potom věrohodnostní funkce
        $$L_1(\lambda) =\prod_{j\in D }f_T(w_j)\prod_{j \in U}R_T(w_j) = \prod_{j\in D }\lambda\e{-\lambda w_j}\prod_{j \in U}\e{-\lambda w_j} = \lambda^{\delta}\e{-\lambda\sum_{i=1}^n w_i}.$$ Provedeme maximálně věrohodný odhad
        $$l_1(\lambda) = \log(L_1(\lambda)) = \delta\ln(\lambda) - \lambda \sum_{i=1}^n w_i$$
        Derivace, položení rovno 0 a po úpravě dostaneme výsledek ve tvaru
        $$\widehat{\lambda}_\mathrm{ML} = \frac{\sum_{j=1}^n \delta_j}{\sum_{i=1}^n w_i} = \frac{\widebar{\delta}_n}{\widebar{w}_n}.$$
        Obdobně můžeme určit také Bayesovský odhad, kde pro $\lambda$ předpokládáme apriorní rozdělení $\lambda\sim \pi(\lambda)$. Vezněme například rozdělení z Conjugated Family exponenciálního rozdělení -- $\pi(\lambda) \sim \mathrm{Gamma}(a,b)$. Potom pro aposteriorní hustotu dostaneme, přičemž opět pro jednoduchost označme $\sum_{i=1}^n w_i  =w$, jako
        $$\pi(\lambda|(\boldsymbol{w},\boldsymbol{\delta})) = \frac{L(\lambda,\boldsymbol{\theta_2}) \pi(\lambda)}{\int L(\lambda,\boldsymbol{\theta_2}) \pi(\lambda)\d \lambda} = \frac{L_1(\lambda)L_2(\boldsymbol{\theta_2}) \pi(\lambda)}{\int L_1(\lambda)L_2(\boldsymbol{\theta_2}) \pi(\lambda)\d \lambda} \sim \mathrm{Gamma}(\delta+a,w+b).$$
        Můžeme tak určit
        $$\widehat{\lambda}_{B} = \EE{\pi(\lambda|(\boldsymbol{w},\boldsymbol{\delta}))} = \frac{a+\delta}{w+b} = \frac{\widebar{\delta}_n + \frac{a}{n}}{\widebar{w}_n + \frac{b}{n}}$$
        Vlastnosti $\widehat{\boldsymbol{\theta}}_1$ závisí na $\boldsymbol{\theta}_2$!
    \end{example}
    \begin{theorem}
        Přibližné rozdělení časového cenzoru $C$ je
        $$ \sqrt{n}\Big(\widehat{\lambda}_\mathrm{ML} - \lambda \Big) \Dto \NN\Big(0, \frac{\lambda^2}{\PP(T \leq C)}\Big)$$
    \end{theorem}
    Odhady $\widehat{\lambda}_{B}$ a $\widehat{\lambda}_\mathrm{ML}$ jsou konzistentní a $\AN$. Naším cílem není ovšem odhadovat parametry. My chceme získat $R_T(t)$ případně $r_T(t)$ při náhodném censoringu. Díky invarianci MLE s tím není problém, například pro exponenciální rozdělení dostaneme
    $$ \widehat{R}_\mathrm{ML}(t) = R(t, \widehat{\bt}_{1,\mathrm{ML}}) = \e{-\widehat{\lambda}_\mathrm{ML} t} = \e{-t\frac{\widebar{\delta}_n}{\widebar{w}_n}}, \qquad \widehat{r}_\mathrm{ML}(t) = \frac{f_T}{R_T} = \frac{\widebar{\delta}_n}{\widebar{w}_n}.$$
    Kdybychom však chtěli získat $\widehat{R}_{B}(t)$, tedy bayesovský odhad, nemůžeme pouze dosadit jako v předchozím případě. Bayesovský odhad totiž není invariantní. Obdobně platí pro $r_B(t)$. Proto tedy
    $$\widehat{R}_{B}(t) \neq  R(t, \widehat{\bt}_{1,B}).$$

    %lecture 12
    \begin{define}
        Mějmě $T\sim \FF_T$ a $C\sim \FF_C$ nezávislé, kde $W = \min(T,C)$ a $\delta = I\lbrace T \leq C \rbrace$. Koziol-Greenův model RC nastává tehdy, pokud existuje konstanta $\gamma >0$, taková že $R_C(t) = (R_T(t))^{\gamma}$ pro $\forall t$. 
    \end{define}

    \begin{theorem}[Koziol-Green, 1992]
        Koziolův-Greenův model nastává právě tehdy když $W$ a $\delta$ jsou nezávislé náhodné veličiny.
    \end{theorem}
    Poďme se nyní podívat jak vypadá intenzita poruch pro Koziolův-Greenův model
    $$ R_C = (R_T(t))^{\gamma} \implies f_C(t) = -R'_C(t) = \gamma f_T(t)(R_T(t))^{\gamma-1}$$
    $$r_C(t) =\frac{f_C}{R_T}=\frac{\gamma f_T(t)(R_T(t))^{\gamma-1}}{(R_T(t))^{\gamma}} = \gamma \frac{f_T}{R_T} = \gamma r_T(t) \implies r_T(t) = \frac{r_C(t)}{\gamma}$$
    Jedná se o speciální případ obecného modelu proporcionálních rizik. Model proporcionálních rizik je také často nazýván Coxovým modelem, či Lehmannovou rodinou.  Uvidíme později.

\section{Odhady RC neparametricky}

    Jako v přechozích případech máme $T$, $C$, $W=\min(T,C)=T^{*}$. Hvězdičku budeme pro jednoduchost vynechávat. Naměříme experiment s nějakými censorovanými daty, máme tudíž $(W_j,\delta_j)_{j=1}^n$, plnohodnotně ovšem stačí psát $(T_j,\delta_j)_{j=1}^n$. Ten uspořádáme podle $T_j$, čímž dostaneme $(T_{(j)}, \delta_{(j)}){j=1}^n$. 
    Množinu defektů $D$ již máme zavedenou, zde však potřebujeme $D_t =\lbrace j:\delta_{(j)}=1 \wedge t_{(j)} < t \rbrace$. 
    
    \begin{define}[Risk set]
        Definujeme risk set vztahem 
        $$\mathfrak{R}(t) :=\lbrace j: t_{(j)} \geq t \rbrace.$$ 
        Jsou to tedy indexy těch objektů v experimentu, které ještě nebyly zcensorovány, a které stále fungují a z experimentu nevypadly. Počet prvků $\mathfrak{R}(t)$ označme jako $n_j :=\#\mathfrak{R}(t)$.
    \end{define}
    
    \begin{define}[Kaplanův-Meierův odhad $R_T$, 1958]
        Kaplanův-Meierův odhad spolehlivostní funkce je roven 
        $$\widehat{R}_\mathrm{KM}(t) = \prod_{j\in D_t} \frac{n_j -1}{n_j} = \prod_{j\in \widehat{n}} \widehat{p}_j = \prod_{j\in D_t} \frac{n-j}{n-j+1},\quad \quad  \widehat{p}_j = \begin{cases}
            \frac{n_j - 1}{n_j},& j\in D_t \\1, & j\notin D_t \end{cases}. $$
    \end{define}
    Dá se vykreslit také Kaplan-Meier plot $$ \lbrace t_{(j)}, \widehat{R}_\mathrm{KM}(t_{(j)}) \rbrace_{j=1}^n .$$ Pokusme se nyní Kaplan-Meierův odhad odvodit.
    Interval $(0,t),~t>0$ rozdělme pomocí bodů $u_0, u_1, \dots u_m$.

    Spolehlivostní funkci $R_T(t)$ tak můžeme pomocí věty o součinové pravděpodobnosti rozepsat jako
    $$ R_T(t) = \PP(T \geq t) = \underbrace{\PP(T > u_0)}_{\text{=1}} \underbrace{\PP(T > u_1|T>u_0)}_{\text{= $p_{1}$}}\cdots\underbrace{\PP(T > t|T>u_m)}_{\text{= $p_{m+1}$}} = \prod_{j=1}^{m+1} p_j.$$
    Předpokládáme navíc, že dělení intervalu $(0,t)$ je tak husté, že v jednom podintervalu $(u_k , u_{k+1})$ může nastat pouze jeden censor, jedna porucha, nebo nic. Nenastávají totiž dublovaná $T$ či $C$. Potom lze odhad $p_j$ vyjádřit jako
    $$\widehat{p}_j = \widehat{\PP}(T>u_j|T > u_{j-1}) = \begin{cases}
        1 & \text{nestalo~se~nic}, \\1 & \mathrm{nastal~1~censor}, \\ \frac{n_j -1}{n_j}&  \text{nastala~1~porucha}. \end{cases} $$
    Zmíníme také vlastnosti KM odhadu, přičemž zejména 3. vlasnost je pro nás hodně důležitá.
    
    \begin{theorem}[Vlastnosti KM odhadu]~
        \begin{enumerate}
            \item $\widehat{R}_\mathrm{KM}(t)$ je \textbf{neparametrickým} MLE odhadem $R_T(t)$,
            \item $\widehat{R}_\mathrm{KM}(t)$ je \textbf{konzistentním} odhadem $R_T(t)$,
            \item $\widehat{R}_\mathrm{KM}(t)$ je $\AN\Big(R_T(t), \frac{1}{n}R_T(t)^2 \int_0^t \frac{\d F_u(u)}{(1-H(u))^2} \Big)$, kde $F_u(t) = \PP(T\leq t,\delta=1)$ a $H(t) = \PP(T < t)$.
        \end{enumerate}
    \end{theorem}
    
    Výraz $\frac{1}{n}R_T(t)^2 \int_0^t \frac{\d F_u(u)}{(1-H(u))^2}$ je obtížné vypočítat, a proto ho budeme chtít odhadnout v bodech $t_{(j)}$. Připravíme si odhady
    $$ 1 - \widehat{H}(t_{(j)}) = 1 -\frac{j}{n} = \frac{n-j}{n}, \qquad 1 - \widehat{H}(t_{(j)}^{-}) = \frac{n-j+1}{n},$$
    $$ \big(1 - \widehat{H}(t_{(j)})\big)^2 = \big(1 - \widehat{H}(t_{(j)})\big)(1 - \widehat{H}(t_{(j)}^{-})\big).$$ 
    Funkci $F_u(u)$ aproximujeme empiricky.
    
    \begin{theorem}[Greenwoodova formule]
        Asymptotický rozptyl $\widehat{R}_\mathrm{KM}(t)$ můžeme odhadnout pomocí Greenwoodovy formule jako
        $$ \widehat{\mathrm{AsVar}}(\widehat{R}_\mathrm{KM}(t)) = \frac{1}{n}\widehat{R}_\mathrm{KM}(t)^2 \sum_{j\in D_t} \frac{\frac{1}{n}}{\frac{n-j}{n}\frac{n-j+1}{n}} = \widehat{R}_\mathrm{KM}(t)^2 \sum_{j \in D_t} \frac{1}{n_j(n_j -1)},$$
        kde $n_j = n - j +1$.
    \end{theorem}
    
    Díky odhadu asymptotického rozptylu, který již nezávisí na neznámé distribuci, můžeme asymptotické rozdělení $\widehat{R}_\mathrm{KM}(t)$ napsat ve tvaru
    $$\widehat{R}_\mathrm{KM}(t) \sim \AN\Big(R_T(t), \widehat{R}_\mathrm{KM}(t)^2 \sum_{j \in D_t} \frac{1}{n_j(n_j -1)} \Big).$$
    Dále můžeme dostat interval spolehlivosti pomocí kvantilu normálního rozdělení $u_{1-\frac{\alpha}{2}}$ pro $R_T(t)$ pomocí odhadu $\widehat{R}_\mathrm{KM}(t)$. Kumulativní intenzita poruch $\widehat{\Lambda}_T(t)$ se dá odhadnout jako
    $$\widehat{\Lambda}_\mathrm{KM}(t)\equal{MLE~inv.} = -\ln \widehat{R}_\mathrm{KM}(t) = -\ln \prod_{j\in D_t} \widehat{p}_j = - \sum_{j \in D_t} \ln\Big(1 - \frac{1}{n_j}\Big).$$
    Dále využijeme rozvoje logaritmu $\ln(1-x) = -\Big(x + \frac{x^2}{2} +\frac{x^3}{3}+\dots\Big)$
    a dostaneme 
    $$ \widehat{\Lambda}_\mathrm{KM}(t) = - \sum_{j \in D_t} \ln\Big(1 - \frac{1}{n_j}\Big) = \sum_{j \in D_t} \Big(\frac{1}{n_j} + \frac{1}{2n_j^2} + \frac{1}{3n_j^3} + \dots \Big).$$
    
    \begin{theorem}[Nelson--Aalenův odhad]
        Po zanedbání všech členů kromě prvního dostaneme \textbf{Nelsonův--Aalenův} odhad kumulativní intenzity poruch 
        $$\widehat{\Lambda}_\mathrm{NA}(t)  =\sum_{j\in D_t} \frac{1}{n_j} = \sum_{j \in D_t} \frac{1}{n-j+1}.$$
        Díky tomuto odhad tak můžeme pokračovat v určení spolehlivostní funkce 
        $$ \widehat{R}_\mathrm{NA}(t) =\e{-\widehat{\Lambda}_\mathrm{NA}(t)}. $$
    \end{theorem}
    
    \begin{remark}
        Přestože se může zdát, že se jedná o příliš hrubou aproximaci, bylo zjištěno. že MSE $\widehat{R}_\mathrm{NA}(t)$ může být v některých případech lepší než MSE $\widehat{R}_\mathrm{KM}(t)$.
    \end{remark}
    
    \begin{define}
        Graf $\lbrace t_{(j)}, \widehat{\Lambda}_\mathrm{NA}(t) \rbrace_{j\in D}$ nazveme Nelsonovým-Aalenovým plotem. 
    \end{define}
        
    \begin{remark}[Využití NA odhadu]
        Pomocí NA--plotu lze detekovat distribuční rodinu $F_T$. Pomocí $\widehat{R}_\mathrm{NA}(t)$ lze také určit mediánová doba života $\widehat{t}_{\mathrm{med}}$ nebo střední doba života $$\widehat{\EE{T}} =\widehat{\MTTF} = \int_0^{\infty} \widehat{R}_\mathrm{NA}(t) \d t .$$ Pro odhad intenzity poruch $\widehat{r}_T(t)$ potřebujeme však mít k dispozici například histogramové či jádrové odhady $\widehat{f}_T^{(H)} ,\widehat{f}_T^{(K)}$ upravené pro cenzorovaná data. Nakonec lze pomocí $\widehat{R}_\mathrm{NA}(t)$ navíc také vykreslit QQ--plot, PP--plot, případně TTT--plot (viz příští kapitola).
    \end{remark}
    
    \paragraph{Fitování SKE modelu} Máme k dispozici RC data $(t_{(j)}, \delta_{(j)})_{j=1}^n$
    
    \begin{enumerate}
        \item Zvolíme rozdělení $\FF$, například $\Weib(\alpha, \lambda)$.
        \item Určíme odhady $\widehat{\alpha}, \widehat{\lambda}$ např. MLE, ale pozor! vše pro RC $\longrightarrow \Weib(\widehat{\alpha},\widehat{\lambda} )$.
        \item Vykreslíme spojitou křivku $\widehat{\Lambda}_W$ našeho odhadnutého Weibulovského modelu.
        \item Do stejného plotu vykreslíme $\widehat{\Lambda}_\mathrm{NA}(t)$ (případně $\widehat{\Lambda}_\mathrm{KM}(t)$) a porovnáme je mezi sebou.
    \end{enumerate}
    
    \begin{remark}[Log--Rank Test]
        Test homogenity rozložení dat 
        $$\hypothesis{R_1(t) = R_2(t)}{R_1(t) \neq R_2(t)}$$
        kde $R_1(t)$ může být spolehlivostní funcke pacientů s placebem a $R_2(t)$ totéž akorát pro pacienty s lékem. Test vedoucí na $\chi^2(1)$. Kdyby se testovalo $I$ skupin, pak by test vedl na $\chi^2(I-1)$.
    \end{remark}
    
    \paragraph{Log--Log R plot}
    
    \begin{define}
        Graf $\lbrace t, -\ln\Lambda_{T}(t) \rbrace_{t\in \R^{+}} = \lbrace t, -\ln\ln R_{T}(t) \rbrace_{t\in \R^{+}}$ nazveme Log--Log R plotem. 
    \end{define}
    
    V případě že $R_T = R_0^{\gamma}$ (tedy Koziolův Greenův model) platí, že
    $$-\ln\ln R_{T}(t) = -\ln\gamma -  \ln\ln R_0,$$
    což značí, že průběhy křivek jsou totožné, jenom posunuté o konstantu $-\ln \gamma$.
