\chapter{[SKE] Spolehlivost komponentních systémů, redundantní struktury, důležitost komponent.}

% lecture 7
\subsection{Komponentní systémy}
\begin{define}
	Předpokládejme, že máme systém o $n$ komponentách. Budeme charakterizovat stav $i$-té komponenty $x_i$ binárně jako funguje/nefunguje, tedy $x_i\in\{0,1\}$. Označme $\textbf{x}=(x_1,...,x_n)$ jako \textbf{stavový vektor}. 
	
	Předpokládejme, že známe-li $\textbf{x}$, pak známe stav celého systému $S$, tzn. jsme schopni definovat \textbf{strukturní funkci} $\varphi(\textbf{x})=\begin{cases}
	1,&\text{systém funguje,}\\0,&\text{systém nefunguje.}
	\end{cases}$
\end{define}

Tímto způsobem můžeme zkoumat různě zapojené součásti, třeba sériově, paralelně (jsou tedy v redundanci) apod. U sériového zapojení stačí jedna vadná součástka, aby systém přestal fungovat a u paralelního systému stačí, když alespoń jedna součástka funguje, aby systém fungoval. Pozor ale že záleží na reprezentaci, někdy se např. sériově zapojené součástky chovají, jako by byly paralelní (např. ventily na potrubí, aby voda netekla, alespoň jeden ventil musí fungovat).

\begin{corollary}
	Logické zapojení komponent se může zakreslit pomocí blokových diagramů (RBD - Reliability Block Diagram) (podobné elektrickému zapojení). Pro sériově zapojené součásti máme 
$$ \varphi(\textbf{x})=\prod_{i=1}^{n}x_i=\min(x_i)_1^n.$$
Paralelně zapojené součástky pak
$$ \varphi(\textbf{x})=1-\prod_{i=1}^{n}(1-x_i)\equal{ozn}=\bigsqcup_1^n x_i=\max(x_i)_1^n, $$ kde $\bigsqcup$ označuje inverzní produkt.
\end{corollary}

\begin{corollary}
	k-o-o-n struktura (k-out-of-n) je taková, že systém funguje, pokud je alespoň $k$ z komponent je funkční. Potom
	$$ \varphi(\textbf{x})=\begin{cases}
	1,&\text{pokud }\sumin x_i\geq k,\\0,& \text{jinak.}
	\end{cases}$$
	Tento systém se může použít třeba pro přivolávání hasišů, pokud 2oon zaznamenají kouř. Tímto se eliminují false positive chyby ve stylu zapálená cigareta apod. Například systém 2oo3 se dá zapsat jako paralelní vlákna $\{(1,2), (2,3),(1,3)\}$. Není to však jediná možnost, viz přednáška 7, 34 minuta. OBRÁZEK!!
	
	Máme tedy $\varphi(\textbf{x})=x_1x_2\bigsqcup x_1x_3\bigsqcup x_2x_3$
\end{corollary}

\begin{theorem}
	Mějme systém S o $n$ komponentech. Mějme dále minimální cesty $P_1,...,P_n$ (z tvrzení výše). Pak $$ \varphi(\textbf{x})=\bigsqcup_{j=1}^k \prod_{i\in P_j} x_i.$$
	Naopak z minimálních řezů $K_1,...,K_l$ můžeme dostat
	$$ \varphi(\textbf{x})= \prod_{j=1}^k \bigsqcup_{i\in K_j} x_i.$$
	V praxi pak používáme ten vzorec, který nejvíce zjednodušuje výpočet.
\end{theorem}

\begin{corollary}
	Definujeme strukturu BRIDGE (7-50min). Tady máme 4 cesty, kudy projít, tedy $\{(1,2),(4,5),(1,3,5),(4,3,2)\}$. Z toho můžeme udělat $\varphi(\textbf{x})$. Poměrně jednoduše se dá udělat i diagram řezů.
\end{corollary}

\begin{theorem}[Pivotální rozklad]
	Mějme systém $S$ o $n$ komponentách. Pak můžeme $\varphi(\textbf{x})$ rozepsat jako $\varphi(\textbf{x})=x_i\varphi(1_i,\textbf{x}_{-i})+(1-x_i)\varphi(\textbf{x}_{-i})$, kde fixujeme, že $i$-tý prvek funguje/nefunguje. Tento rozklad se může použít rekurentně až do $\varphi(\textbf{x})=\sum_{\textbf{y}}\prod_{j=1}^n x_j^{y_j}(1-x_j)^{1-y_j}\varphi(\textbf{y})$, kde $\textbf{y}$ jsou $n$-rozměrné binární vektory, kterých je $2^n$.
\end{theorem}

Toto se hodí např. pro BRIDGE, kde je ideální udělat rozklad podle prvku 3. Potom dostaneme 
$$ \varphi(\textbf{x})=x_3\varphi(1_3,\textbf{x}_{-3})+(1-x_3)\varphi(0_3,\textbf{x}_{-3}). $$ 
Tím se výrazně zjednoduší struktura systému.

\begin{define}[Strukturní důležitost komponent]
	Stavový vektor $(1_i,\textbf{x})$ se nazývá \textbf{kritická cesta/vektor pro $i$-tou komponentu}, pokud $\varphi(1_i,\textbf{x})=1\wedge \varphi(0_i,\textbf{x})=0 $, tzn. $\varphi(1_i,\textbf{x})-\varphi(0_i,\textbf{x})=1$.
	
	Označme $\nu_\varphi(i)=\sum_{(\cdot_i,\textbf{x}_{-i})} \varphi(1_i,\textbf{x})-\varphi(0_i,\textbf{x})=1$ počet kritických vektorů pro $i$-tou komponentu.
	
	Číslo $B_\varphi(i)=\frac{\nu_\varphi(i)}{2^{n-1}}$ nazveme \textbf{Birmbaumova míra (strukturní) důležitosti} $i$-té komponenty v $S$.
\end{define}

\begin{example}
	1 a sérii s paralelní 2 a 3. Pak dostaneme
	$$\begin{array}{c|c}
		(\cdot,x_2,x_3) & \varphi(1_i,\textbf{x})-\varphi(0_i,\textbf{x})=1 \\\hline
		\cdot\,00 & 0 \\
		\cdot\,01 & 1 \\
		\cdot\,10 & 1 \\
		\cdot\,11 & 1 
	\end{array}.$$
	Jelikož $2^{n-1}=4$, pak $\nu_\varphi(1)=3$, takže $B_\varphi(1)=\frac{3}{4}$ apod.
	
\end{example}


\begin{define}
	$i$-tá komponenta se nazývá \textbf{irelevantní}, pokud $\varphi(1_i,\textbf{x})=\varphi(0_i,\textbf{x})$, $\forall\textbf{x}_{-i}$. Tedy že na jejím vypnutí/zapnutí nezáleží. Příkladem je záznamové zařízení při požárním poplachu, které je sice důležité, ale na samotný poplach nemá vliv.
\end{define}

\begin{define}
	Systém komponent se nazývá \textbf{koherentní}, pokud neobsahuje irelevantní komponentyf a strukturní funkce je neklesající v každé své proměnné.
\end{define}

\begin{theorem}
	Pro $S$ koherentní platí, že $\varphi(\textbf{0})=0$, $\varphi(\textbf{1})=1$ a $$ \prod_{i=1}^n x_i\leq \varphi(\textbf{x})\leq \bigsqcup_{i=1}^n x_i. $$
\end{theorem}

\begin{corollary}
	Pro výpočet $R_S(t)$ využijeme čas $t$ a ztotožníme poruchovostní komponenty $X_i(t)=\begin{cases}
1,&\text{s pravděpodobností }p_i=\PP(T_i>t),\\0,&\text{jinak.}
\end{cases}$ Dosazením do strukturní funkce potom vyrobíme $\varphi\big(\textbf{X}(t)\big)$, kde $\textbf{X}(t)$ je vektor veličin $X_i(t)$, které charakterizují funkčnost/nefunkčnost komponenty v čase $t$. Potom
$$ R_S(t)=\PP(T_S>t)=\PP\big(\varphi\big(\textbf{X}(t)\big)=1\big)=1\cdot\PP\big(\varphi\big(\textbf{X}(t)\big)=1\big)+0\cdot\PP\big(\varphi\big(\textbf{X}(t)\big)=0\big)=\EE{\varphi\big(\textbf{X}(t)\big)}. $$
\end{corollary}

\begin{example}
	Pro sériově zapojený $S$ dostaneme
	$$ R_S(t)=\E \varphi\big(\textbf{X}(t)\big)\equal{id}\prod_{i=1}^n \E X_i(t)=\prod_{i=1}^n R_i(t)=\prod_{i=1}^n \e{-\int_{0}^{t}\lambda_i(\tilde{t})\d\tilde{t}}=\e{-\int\limits_{0}^t\sumin \lambda_i(\tilde{t})\d\tilde{t}}, $$
	kde $\sumin \lambda_i(\tilde{t})\d\tilde{t}=\lambda_S(\tilde{t})$. Z toho vyplývá, že např. pro 100 komponent o $R_i(t)=0.995$ dostaneme systém o spolehlivosti $R_s(t)=0.606$. Sériové systémy jsou tedy nejhorší možné, co se týče spolehlivosti. 
\end{example}

\begin{example}
	Pomocí tří imaginárních komponent (DFR,CFR,IFR) jde modelovat fyzická (reálná) komponenta s vanovitým průběhem FR. U paralelních součástí máme naopak 
	$$ R_S(t)=\E\varphi\big(\textbf{X}(t)\big)=\E\Big[\bigsqcup_{i=1}^n X_i(t)\Big]=\E\Big[1-\prod_{i=1}^n(1-X_i(t))\Big]\equal{id}1-\prod_{i=1}\big(1-R_i(t)\big)=\bigsqcup_{i=1}^n R_i(t). $$
\end{example}

\begin{example}
S je k-o-o-n: 
$$ R_S(t)=\E\varphi\big(\textbf{X}(t)\big)= \PP\big(\sum_{i=1}^n X_i(t)\geq k\big)\equal{iid}\sum_{x=k}^n {n\choose x} R(t)^{x} (1-R(t))^{n-x} = \sum_{x=k}^n {n\choose x} \e{-\lambda tx} \big(1-\e{-\lambda t}\big)^{n-x}   $$
Z čehož můžeme potom získat střední dobu do poruchy
$$\MTTF_S = \sum_{x=k}^n {n\choose x} \int_0^\infty \e{-\lambda tx} \big(1-\e{-\lambda t}\big)^{n-x} \d t = \Bigg| \e{-\lambda t} = y \Bigg| = \dots =  \frac{1}{\lambda} \sum_{x=k}^n \frac{1}{x},$$
kde jsme využili substice, která integrál převede na beta funkci a dále se dá zjednodušit do výše uvedeného tvaru. V tabulce jsou uvedeny konkrétní případy $\MTTF$ pro různá $k$ či $n$.

\begin{table}[h]
	\centering
	\begin{tabular}{|l|l|l|l|l|}
		\hline
		$k$\textbackslash $n$& 1 & 2 &3  & 4 \\ \hline\hline
		1&$\frac{1}{\lambda}$  &  $\frac{3}{2}$$\frac{1}{\lambda}$& $\frac{11}{6}$$\frac{1}{\lambda}$ & $\frac{25}{12}$$\frac{1}{\lambda}$ \\ \hline
		2& $\times$ & $\frac{1}{2}$$\frac{1}{\lambda}$ & $\frac{5}{6}$$\frac{1}{\lambda}$ &  \\ \hline
		3&  &  & $\frac{1}{3}$$\frac{1}{\lambda}$ &  \\ \hline
		4&  &  &  & $\frac{1}{4}$$\frac{1}{\lambda}$ \\ \hline
	\end{tabular}
\end{table}
\end{example}

\begin{theorem}
	Mějme $n$-komponentní systém $S$ a stavové vektory $\textbf{x}$, $\textbf{y}$. Potom pro koherentní strukturu S platí $\varphi(\textbf{x} \bigsqcup \textbf{y}) \geq  \varphi(\textbf{x}) \bigsqcup \varphi(\textbf{y})$.
\end{theorem}

%lecture 9

\subsection{Redundance}
Podívejme se, jak můžeme zvolit redundanci prvků v systému.
	\begin{enumerate}
		\item \textbf{aktivní} - pararelní záloha do plného provozu (rezervní komponenta se může porouchat)
		\item \textbf{pasivní} - rezervní komponenta čeká ve stand-by režimu  (rezervní komponenta se nemůže porouchat)
		\item \textbf{částečná} - rezervní komponenta je pod minimálním zatížením (rezervní komponenta se může porouchat)
		\end{enumerate}
Jak vyřešit tyto systémy? 
	\begin{enumerate}
	\item \textbf{aktivní} - paralelní systémy, které jsme doposud dělali, můžeme spočítat $R_S, \lambda_S , \MTTF_S$
	\item \textbf{pasivní} - máme switch, který musí být schopný zařadit do systému rezervní komponentu 
	\begin{enumerate}
		\item \textbf{pasivní + perfect switch} - $T_S = \sum_{i=1}^n T_i$, $\MTTF = \sum_{i=1}^n \MTTF_i$, $R_S(t) = \PP\Big(\sum_{i=1}^n T_i\Big) = \otimes_{i=1}^n R_i(t)$ a pro dostatečně velké $n$ je $\sum_{i=1}^n T_i $ asymptoticky normální nebo $AG_{\alpha}$.
		\item \textbf{pasivní + imperfect switch}  -  charakterizujeme switch pomocí pravděpodobnosti, tedy pravděpodobnost, že switch skutečně přepne na záložní komponentu, je 1-$p$. Předpokládáme navíc, že komponenty včetně switche jsou nezávisle fungující. 
		\begin{enumerate}
			\item Komponenta 1 drží až do času t, to nastane s pravděpodobností $R_1(t)$.
			\item Komponenta 1 se porouchala v určitém čase v $(0,t)$, když označíme $\tau < t$ jako čas poruchy,tak se komponenta 1 porouchala v $(\tau,\tau + \d\tau)$. Pravděpodobnost, že nastane tato situace se dá jednoduše zapsat jako hustota $f_{T_1}(\tau)\d\tau$ $\longrightarrow$ Switch přepne s pravděpodobností 1-$p$ v čase $\tau$ $\longrightarrow$ Komponenta 2 je nyní aktivní v čase $\tau$ a funguje až do $t$. Toto nastane s pravděpodobností $R_2(t-\tau)$.
	\end{enumerate}
Podívejme se nyní na pravděpodobnost, že je celý tento systém $S$ funkční.
\begin{align*}
 \PP\Big(S\mathrm{~je~OK~v~case~}t \Big) =  & \PP(T_S > t ) = R_S(t) = \PP(\mathrm{i.}) + \PP(\mathrm{ii.}) = \\ &
  R_1(t) + \int_0^t (1-p)R_2(t-\tau) f_{T_1}(\tau)\d\tau = \\ &
 \e{-\lambda_1 t} + \int_0^t (1-p)\e{-\lambda_2 (t-\tau)} \lambda_1\e{-\lambda_1 \tau}  \d\tau = \\ &
 \e{-\lambda_1 t} + \frac{(1-p)\lambda_1}{\lambda_1-\lambda2}\e{-\lambda_2 t} - \frac{(1-p)\lambda_1}{\lambda_1-\lambda2}\e{-\lambda_1 t}
 \end{align*}
Navíc můžeme také určit 
\begin{align*}
	\MTTF_S = \int_0^{\infty} R_S(t) \d t = \frac{1}{\lambda_1} + \frac{1-p}{\lambda_2}
	\end{align*}
	\end{enumerate}
	\item \textbf{částečná + imperfect switch} - Máme $T_1$ se spolehlivostí $R_1$, $T_2$ se spolehlivostí $R_2$ a switch, který přepíná s pravděpodobností $(1-p)$. Navíc máme $T_0 \sim R_0$, což je doba do poruchy komponenty 2 v době minimálního zatížení. Obdobně lze rozdělit na dva případy.
		\begin{enumerate}
			\item Komponenta 1 drží až do času t, to nastane s pravděpodobností $R_1(t)$.
			\item Komponenta 1 se porouchala v určitém čase v $(0,t)$, když označíme $\tau < t$ jako čas poruchy,tak se komponenta 1 porouchala v $(\tau,\tau + \d\tau)$. Pravděpodobnost, že nastane tato situace se dá jednoduše zapsat jako hustota $f_{T_1}(\tau)\d\tau$ $\longrightarrow$ Switch přepne s pravděpodobností 1-$p$ v čase $\tau$ $\longrightarrow$ Komponenta 2 není porouchána v čase $\tau$, nastává s pravděpodobností $R_0(\tau)$ $\longrightarrow$ Komponenta 2 je funkční v čase $t$, tzn. $R_2(t-\tau)$.
		\end{enumerate}
	
$$\Huge{OBR!!Nejlepe~dva~obrazky~vedle~sebe~aktivni~zaloha~imperfect~switch}	$$

	Máme tudíž o jednu variantu navíc, než v předchozím případě. Opět lze určit 
	\begin{align*}
		 R_S(t) = &	R_1(t) + \int_0^t (1-p)R_0(\tau)R_2(t-\tau) f_{T_1}(\tau)\d\tau = \\ &
		\e{-\lambda_1 t} + \frac{(1-p)\lambda_1}{\lambda_0+\lambda_1-\lambda_2}\big(\e{-\lambda_2 t} - \e{-(\lambda_0 + \lambda_1) t} \big)
	\end{align*}
\begin{align*}
	\MTTF_S = \int_0^{\infty} R_S(t) \d t = \frac{1}{\lambda_1} + \frac{1-p}{\lambda_2} - (1-p)\frac{\lambda_0}{\lambda_2(\lambda_1 + \lambda_0)}
\end{align*}
\end{enumerate}



