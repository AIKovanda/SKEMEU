\chapter{[SKE] TTT transformace a plot, optimální preventivní údržba, model proporcionálních rizik.}

\subsection{TTT transformace}
\begin{define}[TTT transformace]
Mějme veličinu $T\sim F$ takovou, že $\mu<+\infty$. Definujeme TTT transformaci vztahem $$ \HF(v)=\int_{0}^{F^{-1}(v)}\big(1-\RT(t)\big)\d t,\quad v\in[0,1],\quad F^{-1}(0)=0,\quad F^{-1}(1)=x_F,$$
kde $x_F$ je pravý krajní bod rozdělení $F$. Definujeme dále škálovanou TTT transformaci jako $\phi_F(v)=\frac{\HF(v)}{\mu}$.
\end{define}

% lecture 6

\begin{theorem}
Mějme $F$ spojitou a ostře rostoucí. Pak pro $\forall t\in\big(0,F^{-1}(1)\big)$ platí, že $$\HF\big(F(t)\big)=\frac{1}{r_T(t)}.$$
\end{theorem}

\begin{define}[Empirická TTT transformace]
	Pro $\textbf{T}:=(T_1,...,T_n)~iid~F$ definujeme \textbf{Empirickou TTT transformaci} jako
	$$ H_n^{-1}(v):=\int_{0}^{F_n^{-1}(v)}\big(1-F_n(t)\big)\d t,\quad \phi_n(v)=\frac{H_n^{-1}(v)}{H_n^{-1}(1)}. $$
	Platí, že $\inv{H_n}\rightrightarrows \inv{H_F}(v)$.
\end{define}

\begin{define}[TTT plot]
	Definujeme \textbf{TTT plot} jako graf $\Big\{\frac{i}{n},\phi_n\big(\frac{i}{n}\big)\Big\}_0^n$.
\end{define}

TTT plot se tak jmenuje, protože $\inv{H_n}\big(\frac{i}{n}\equal{...}\frac{1}{n}\Big[\sum_{j=1}^i t_{(j)}+(n-i)t_{(i)}\Big]$, kde obsahu hranaté závorky se říká \textbf{Total test in time} v bodě $t_{(i)}$ (od $i$-té součástky počítáme pouze $t_{(i)}$, protože všechny byly v testu po dobu $t_{(i)}$). Test nutně nemusí skončit v době poruchy některé součástky, a tedy můžeme zavést symbol $\tau(t):=\sum_{j=1}^it_{(j)}+(n-i)t$, kde $T_{(i)}<t<T_{(i+1)}$.

\subsection{Preventivní údržba}
Představme si třeba převodovku automobilu, kde jedna z komponent, klínový řemen, sice stojí pár korun ($c$), ale porucha řemene kompletně zničí celou převodovku, což se může prodražit ($k$). V této části se tedy pokusíme optimalizovat maintenance policy tak, aby se snížily náklady na provoz automobilu. Ještě vyšší potřeba je kontrola komponent u jaderných elektráren, kde je potřeba dělat odstávky a vyměnit všechny potřebné součásti tak, aby se minimalizovala pravděpodobnost havárie.

Celkové náklady v čase $t$ se tedy spočítají jako $\sum_{\text{plán}}c+\sum_{\text{poruchy}}(c+k)$. Střední náklady na výměnnou periodu jsou tedy rovny $c+k\PP(Tyleq t_0)$. Definujeme \textbf{střední náklady za jednotku času} $C_A(t_0)=\frac{c+k\PP(T\leq t_0)}{\mathrm{MTBR}(t_0)}$ pro Mean Time Before Replacement ve tvaru \[
\begin{split}
	\mathrm{MTBR}(t_0)&=\E\bigg(Z(T)=\begin{cases}
		T,& T\leq t_0,\\ t_0,& T>t_0
	\end{cases}\bigg)=\int_{0}^{t_0}tf_T\d t+\int_{t_0}^{+\infty}t_0 f_T(t)\d t\equal{...} \\ &= \int_{0}^{t_0\stackrel{!}{=}\inv{F}(v_0)}\big(1-F(t)\big)\d t=\HF(v_0) 
\end{split}
\]  a chceme $t_0$ optimalizovat tak, aby se minimalizovala veličina $$C_A(v_0)=\frac{c+k\overbrace{F(t_0)}^{v_0}}{\HF(v_0)}=\frac{1}{\mu}\frac{c+kv_0}{\phi_F(v_0)}.$$

Po derivaci úpravou dostaneme 
$$ \frac{\phi_F'}{\phi_F}=(\ln \phi_F)'=\frac{k}{c+kv_0}.$$

Vyřešením této diferenciální rovnice dostaneme optimální $v_0$ (ne vždy ale máme $\phi_F$ a jsou tam i další problémy).

\subsection{Coxův Model Proporcionálních rizik}
Doba do poruchy $T$ je ovlivněna doprovodnými vysvětlujícími faktory $\bold{X} =  (\bold{X}_1, \bold{X}_2,\dots, \bold{X}_q)$. Můžou to být kategorické proměnné - kouří , nekouří - nebo se může jednat i o spojitou proměnnou.
$$ r_T(t,\bold{X}) = r_0(t) \cdot \psi(\bold{X}, \boldsymbol{\beta})$$
Dosadíme za $\psi(\bold{X}, \boldsymbol{\beta}) = \e{\boldsymbol{\beta}\bold{X}} $ a zlogaritmujeme - získáme tak zobecněnou regresní úlohu.
$$\ln r(t) = \ln r_0(t) + \boldsymbol{\beta}\bold{X} $$
Snažíme se namodelovat výše zmíněnou závislost, odhadujeme regresní parametry. Ale pozor, tyto parametry odhadujeme zapřítomnosti RC censorovaných dat. Nezapoměňme, že naše forma dat je $(T_j, \delta_j)_{j=1}^n$. Můžeme definovat hazard ratio
$$ \mathrm{HR(\bold{X}_1, \bold{X}_2)} = \frac{r_T(t, \bold{X}_1)}{r_T(t, \bold{X}_2)} = \frac{\psi_1}{\psi_2} = \e{\sum_{k=1}^{q} \beta_k(X_k^{(1)}- X_k^{(2)})} = \e{\beta_1}$$
Pro odhad parametrů $\boldsymbol{\beta}$ se používá tzv. částečný věrohodnostní odhad (partial ML)
$$L_p(\boldsymbol{\beta})  =\prod_{j\in D } \frac{\psi(\bold{X}_{(j)}, \boldsymbol{\beta})}{\sum_{k \in \mathfrak{R}(t_{(j)})}\psi(\bold{X}_{(k)}, \boldsymbol{\beta})},$$
ze kterého pak získáme
$$ \hat{\boldsymbol{\beta}}_{PML} = \argmax L_p(\boldsymbol{\beta}).$$


